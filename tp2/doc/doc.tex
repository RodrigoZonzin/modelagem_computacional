\documentclass[]{article}
\usepackage[portuguese]{babel}
\usepackage[letterpaper,top=2cm,bottom=2cm,left=3cm,right=3cm,marginparwidth=1.75cm]{geometry}
\usepackage{esdiff}
\usepackage{amsmath}
\usepackage{graphicx}
\usepackage{subfigure}
\usepackage{booktabs}
\usepackage{subfigure}
\usepackage{xfrac}
\usepackage[colorlinks=true, allcolors=blue]{hyperref}


%opening
\title{Relatório: Estimando parâmetros do modelo SIR para o caso de São João del-Rei, MG, em 2021}
\author{Rodrigo José Zonzin Esteves}
\date{Dezembro de 2024}

\begin{document}

\maketitle

\section{O Modelo SIR e estratégias adotadas}
O modelo Suscetível-Infectado-Recuperado (SIR) é dado pelo sistema de EDO's apresentado pela Equação \ref{eq01}. Os parâmetros $\alpha$ e $\beta$ podem ser estimados através de um processo de ajuste de curvas não-lineares. 

\begin{equation}	
	\begin{cases}
		
		\diff{S(t)}{t} = -\beta S(t) I(t) \\
		\diff{I(t)}{t} = \beta S(t)  I(t) - \alpha I(t) \\
		\diff{R(t)}{t} = \alpha I(t)
	\end{cases}	
	\label{eq01}
\end{equation}

Para o ajuste de curvas, considerou-se a Norma 2 $||\cdot||_2$ como medida de erro. 
\begin{equation}
	||erro|| = \sqrt{\sum_{i = 1}^{n} (y_i - \hat{y}_i)^2}
\end{equation}

Os dados para o ajuste consistem em uma série temporal  com o número de infectados acumulado diariamente em São João del-Rei. Esse dado é representado pelo vetor $\mathbf{I}$. 

Portanto, para que se estime $\alpha$ e $\beta$, é preciso minimizar a função-objetivo dada pela Equação \ref{eq03}. 

\begin{equation}
	\begin{aligned}
		&\textbf{min} \quad \text{erro}(\alpha, \beta) = \sqrt{\frac{\sum_{i \in \mathbf{I}} (\hat{i} - i)^2}{\sum_{i \in \mathbf{I}} i^2}} \\
		&\alpha, \beta \in [0, 1]
	\end{aligned}
	\label{eq03}
\end{equation}


A otimização da Equação \ref{eq03} foi obtida através do método de Evolução Diferencial. Considerou-se ainda, para cada ajuste, as seguintes estratégias de mutação e cruzamento: 
\begin{enumerate}
	\item rand1bin
	\item rand2bin
	\item best1bin
	\item best2bin
\end{enumerate}

\section{Resultados}
\subsection{Primeiras doze semanas de 2021}

\begin{table}[h!]
	\centering
	\begin{tabular}{l c | c c}
		\toprule
		Estrategia & Erro Associado & $\alpha$ & $\beta$ \\
		\hline
		rand1bin & 0.16339400  & 0.00001169 & 0.83307879 \\
		rand2bin & 0.16156398  & 0.00001189 & 0.84978616 \\
		best1bin & 0.21111678  & 0.00001204	& 0.84661256 \\		
		best2bin & 11.61501726 & 0.02000000 & 0.85000000 \\
		\bottomrule
		
	\end{tabular}
\end{table}

\begin{figure}[h!]
	\centering
	
	\subfigure[Curva I]{\includegraphics[width=0.45\linewidth]{figs/I_rand1bin.png}}
	\subfigure[Todas as Curvas]{\includegraphics[width=0.45\linewidth]{figs/results_rand1bin.png}}
	
	\caption{Resultados para rand1bin}
	\label{res01}
\end{figure}

\begin{figure}[h!]
	\centering
	
	\subfigure[Curva I]{\includegraphics[width=0.45\linewidth]{figs/I_rand2bin.png}}
	\subfigure[Todas as Curvas]{\includegraphics[width=0.45\linewidth]{figs/results_rand2bin.png}}
		
	\caption{Resultados para rand2bin}
	\label{res02}
\end{figure}

\begin{figure}[h!]
	\centering
	
	\subfigure[Curva I]{\includegraphics[width=0.45\linewidth]{figs/I_best1bin.png}}
	\subfigure[Todas as Curvas]{\includegraphics[width=0.45\linewidth]{figs/results_best1bin.png}}
	
	\caption{Resultados para best1bin}
	\label{res03}
\end{figure}

\begin{figure}[h!]
	\centering
	
	\subfigure[Curva I]{\includegraphics[width=0.45\linewidth]{figs/I_best2bin.png}}
	\subfigure[Todas as Curvas]{\includegraphics[width=0.45\linewidth]{figs/results_best2bin.png}}
	
	\caption{Resultados para best2bin}
	\label{res04}
\end{figure}



\newpage

\subsection{Ano de 2021}

\begin{table}[h!]
	\centering
	\begin{tabular}{l c | c c}
		\toprule
		Estrategia & Erro Associado & $\alpha$ & $\beta$ \\
		\hline
		rand1bin & 1.43433162 & 0.02000000 & 0.85000000 \\
		rand2bin & 1.43433162 & 0.02000000 & 0.85000000 \\
		best1bin & 0.89975065 & 0.00001000 & 0.69827012 \\		
		best2bin & 0.89975065 & 0.00001000 & 0.69827012 \\		
		\bottomrule
	\end{tabular}
\end{table}

\begin{figure}[h!]
	\centering
	
	\subfigure[Curva I]{\includegraphics[width=0.45\linewidth]{figs_2021/I_rand1bin.png}}
	\subfigure[Todas as Curvas]{\includegraphics[width=0.45\linewidth]{figs_2021/results_rand1bin.png}}
	
	\caption{Resultados para rand1bin}
	\label{res05}
\end{figure}

\begin{figure}[h!]
	\centering
	
	\subfigure[Curva I]{\includegraphics[width=0.45\linewidth]{figs_2021/I_rand2bin.png}}
	\subfigure[Todas as Curvas]{\includegraphics[width=0.45\linewidth]{figs_2021/results_rand2bin.png}}
	
	\caption{Resultados para rand2bin}
	\label{res06}
\end{figure}

\begin{figure}[h!]
	\centering
	
	\subfigure[Curva I]{\includegraphics[width=0.45\linewidth]{figs_2021/I_best1bin.png}}
	\subfigure[Todas as Curvas]{\includegraphics[width=0.45\linewidth]{figs_2021/results_best1bin.png}}
	
	\caption{Resultados para best1bin}
	\label{res07}
\end{figure}

\begin{figure}[h!]
	\centering
	
	\subfigure[Curva I]{\includegraphics[width=0.45\linewidth]{figs_2021/I_best2bin.png}}
	\subfigure[Todas as Curvas]{\includegraphics[width=0.45\linewidth]{figs_2021/results_best2bin.png}}
	
	\caption{Resultados para best2bin}
	\label{res08}
\end{figure}

\section{Conclusão}
Conforme \cite{Barcelos2011}, ``A estratégia a ser adotada para um problema é determinada por tentativa e erro". Para o caso em tela, observou-se que a estratégia que apresentou o menor erro é foi a rand2bin. 

Para o ano de 2021, observa-se uma subestimação do número de infectados em relação ao dado observado. Isso ocorre pois o modelo SIR não considera a inclusão de recuperados no grupo de Suscetíveis. De fato, já se sabe que a reinfecção por Covid-19  é um fato importante no comportamento epidemiológico da doença \cite{Silva2021}, não sendo mapeado pelo modelo em estudo. 

\bibliographystyle{plain} 
\bibliography{referencias.bib} 

\end{document}
